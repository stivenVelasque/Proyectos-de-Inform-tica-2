\documentclass[12pt]{article}
\usepackage[spanish]{babel}
\usepackage{amsmath}
\usepackage{graphicx}

\begin{document}

\begin{center}
\bf{\sc\Huge Universidad de Antioquia}\\
\end{center}
\vspace{120pt}
\begin{center}
\bf{\sc\Huge Stiven Velásquez López }\\
\end{center}
\vspace{200pt}
\begin{center}
\bf{\sc\Huge Medellín}
\end{center}
\begin{center}
\bf{\sc\Huge 2020}\\
\end{center}\
\newpage


\begin{center}

\bf{\sc\Huge ¿Cómo la crisis de los fundamentos derivó en el nacimiento de la computación moderna?. }\\
\end{center}
\begin{flushleft}
\vspace{8PT}
\large
\end{flushleft}

\section{ INTRODUCCIÓN}
\large

El funcionamiento de la sociedad moderna está estrechamente ligada al nacimiento de los ordenadores. Es importante recordar a informáticos, programadores y profesionales relacionados al área de la computación, que los ordenadores en primera instancia fueron inventados para que ayudasen a aclarar dudas concernientes a los fundamentos de la matemática.

\vspace{10PT}
Muchos estudiantes en escuelas, colegios y universidades se indagan del por qué o para qué les va a servir el temario de matemáticas y todas esas fórmulas aprendidas de memoria. Sin embargo, unos años más tarde, comprenden que el mundo que los rodea no es nada sin las matemáticas y como se dijo anteriormente, los primeros informáticos eran matemáticos que querían automatizar ciertos procesos de cálculo.

\vspace{10PT}
A continuación se presentará la esencia de los argumentos y aportaciones fundamentales  de personajes como  Georg Cantor,Bertrand Russell, Kurt Gödel, Alan Turing y David Hilbert, en los fundamentos matemáticos que ayudaron en el nacimiento de la computación.

\newpage

\section{¿CÓMO LA CRISIS DE LOS FUNDAMENTOS DERIVO EN EL NACIMIENTO DE LA COMPUTACION MODERNA?}
\large

Se empezará reseñando a Georg Cantor, quien desarrollando la Teoria de Conjuntos concluyó que hay infinitos de diversos tamaños;no se trata sólo de lo finito y lo infinito. Por otro lado tenemos a Bertrand Russell, un matemático que halló razonamientos supuestamente impecables, pero con el análisis adecuado llevaban a discordancias o contradicciones.

\vspace{10PT}
David Hilbert tuvo el reto de demostrar que los axiomas de la aritmética no conducen a ninguna contradicción. Su programa comprendía demostrar que todo conocimiento matemático se deriva de un conjunto finito de axiomas, y puede demostrarse que tal sistema axiomático no conduce a contradicciones cuando se derivan conclusiones de ellos, y para cada afirmación de su sistema se podría demostrar su certeza o falsedad.

\vspace{15PT}
Gödel descubrió que Hilbert estaba equivocado; no hay modo de que exista un sistema axiomático para la totalidad de la matemática en el que quede claro si un enunciado es verdadero o no.

\vspace{15PT}
Luego ocurrió un avance de gran importancia, y se le atribuye a Alan Turing, el cual descubrió la no compatibilidad. Hilbert en cierta ocasión había aclarado que existe un procedimiento mecánico que decidía si una demostración se atenía a las reglas que este tenía establecidas o no, y el primero en afianzar el concepto fue Alan Turing, quien estableció que este procedimiento mecánico se traducía en una máquina, la cual llamo ¨Máquina de Turing¨, este mecanismo fue un elemento fundamental en la teoría de la computación,se encargaba del proceso automático para determinar si un problema matemático podía ser resuelto o no mediante un procedimiento definido. Fue ideado para resolver una operación concreta. Este proceso se asemeja a un ordenador gracias a su capacidad de llevar a cabo múltiples procesos.
Turing descifró el lenguaje secreto utilizado por los nazis, conocido como el Código Enigma y contribuyó según historiadores a nada menos que acortar la Segunda Guerra Mundial.
\newpage

\section{ CONCLUSIONES}
\large
\vspace{15PT}

Analizar la historia de la matemática y su ayuda en la computación es un proceso arduo, pasando por la obra de los personajes anteriormente mencionados, yendo desde George Cantor con sus cuestionamientos sobre el infinito y deducciones de  la existencia de infinitos de diferentes tamaños, Bertrand Russell con sus razonamientos que a la larga parecen impecables y lógicos, pero que con una indagación correspondiente se llega a contracciones y paradojas, luego llegamos a David Hilbert quien con sus premisas pretendía demostrar la consistencia y veracidad de los axiomas matemáticos. Luego Godel contradice a Hilbert, afirmando que es imposible tener axiomas matemáticos con los que en la aritmética se pueda decir con certeza si su enunciado es verdadero o no lo es, refutando la consistencia de estos. Llegamos por fin a una eminencia en cuanto a lógica se refiere, y nos referimos a Alan Turing, quien conceptualizando sus argumentos logra realizar una máquina que sistematizaba procedimientos, siendo capaz de llevar a cabo múltiples procesos.

\vspace{15PT}
El mundo cambia, los pensamientos se mueven de manera diferente a medida que avanza el tiempo, es importante analizar acciones pasadas para no caer y recaer en antiguas dificultades. La computación se inicia debido a una crisis en torno a los fundamentos y obra de muchos matemáticos a lo largo del tiempo, es importante decir que todo fue un proceso, cada una de estas personalidades se basaron en el trabajo del anterior. 

\vspace{15PT}
El mérito de que las computadoras realicen diversas labores en un lapso de tiempo no es un  acto que se haga por si solo, los programadores son los que indican que actividades y tareas deben realizarse, siendo trabajos que se repiten de forma reiterada, y se determina de manera acertada que la rama computacional esta íntimamente ligada a la rama de las matemáticas, teniendo cimientos que se adentran en estos grandes pensadores como lo fue Alan Turing quien presentó la formalización de un algoritmo, con límites en lo que puede ser calculado, y un modelo "puramente mecánico" para la computación, y así se inicia el proceso por el cual podemos tener acceso a un ordenador personal, un celular, o cualquier sistema que conlleve una serie de pasos para su funcionamiento
\newpage

\section{ BIBLIOGRAFÍA}
\large
\vspace{15PT}
\begin{thebibliography}{X}
\bibitem{Nel} \textsc{NELO MAESTRE } y \textsc{AGATA TIMÓN},
\textit{.Así terminó el sueño de las matemáticas infalibles (y de paso, nació la computación moderna)}, [Online]. Available: https://www.bbvaopenmind.com/ciencia/matematicas/asi-termino-el-sueno-de-las-matematicas-infalibles/?utm, [Consulta 25 de marzo de 2020].
\bibitem{Jef} \textsc{JEFFREY SHALLIT}(1995),
<<.A Very Brief History of Computer Science>>,
\textit{[Online]. Available:https://cs.uwaterloo.ca/~shallit/Courses/134/history.html}, \textbf{[Consulta 26 de marzo de 2020]}.

\bibitem{Sta} \textsc{STANFORD ENCYCLOPEDIA OF PHILOSOPHY}(May 24, 2019),<<Hilbert’s Program>>,\textit{[Online]. Available:.https://plato.stanford.edu/entries/hilbert-program/}, \textbf{[Consulta 26 de marzo de 2020]}.

\bibitem{Fer} \textsc{FERNANDO CUARTERO}(Marzo 1 de 2012),<<El problema de la decidibilidad. Alan M. Turing III>>,\textit{[Online]. Available:https://www.hablandodeciencia.com/articulos/2012/03/01/el-problema-de-la-decidibilidad-alan-m-turing-iii/}, \textbf{[Consulta 26 de marzo de 2020]}.

\bibitem{J} \textsc{J J O'CONNOR AND E F ROBERTSON}(2003),<<Kurt Gödel>>,\textit{[Online]. Available:http://mathshistory.st-andrews.ac.uk/Biographies/Godel.html}, \textbf{[Consulta 26 de marzo de 2020]}.

\bibitem{J} \textsc{O'CONNOR, JOHN J.; ROBERTSON, EDMUND F}(1998),<<Georg Cantor>>,\textit{[Online]. Available:http://mathshistory.st-andrews.ac.uk/Biographies/Cantor.html}, \textbf{[Consulta 26 de marzo de 2020]}.

\bibitem{J} \textsc{RONALD CLARK}(1976),<<The Life of Bertrend Russell>>.

\end{thebibliography}


\end{document}